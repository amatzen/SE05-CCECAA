\documentclass[11pt]{article}

\usepackage[a4paper, total={6in, 8in}]{geometry}
\usepackage{geometry}
\usepackage{xcolor}
\usepackage{graphicx}
\usepackage{titling}
\usepackage{afterpage}
\usepackage{fontspec}
\usepackage{titlesec}
\usepackage[bottom]{footmisc}
\usepackage{hyperref}
\usepackage{fancyhdr}

\definecolor{titlepagecolor}{rgb}{0,0,.65}

% Fonts

\setmainfont{IBMPlexSans}[
	Path=./fonts/,
	BoldFont=*-Bold,
	UprightFont=*,
	ItalicFont=*-Italic,
	BoldItalicFont=*-BoldItalic,
]

\newfontfamily{\headerfont}{Serenity}[
	Path=./fonts/,
	UprightFont=*-Medium,
]


% Customization


% Actual document
\geometry{
  a4paper,
  left=2.5cm,
  right=2.5cm,
  top=2.15cm,
  bottom=1.15cm
}
\begin{document}

\setlength\parindent{0pt}
\setlength{\parskip}{.15em}
\pagestyle{empty}

\title{Industry Talks \#3: Netcompany}
\author{Alexander Matzen \addvspace{1em} Student Number: 493840 \newline E-mail Address: almat20@student.sdu.dk}
\date{\today}

\begin{titlepage}
\newgeometry{left=3cm,top=3cm, bottom=3cm}
\pagecolor{titlepagecolor}\afterpage{\nopagecolor}
\noindent
\color{white}
\makebox[0pt][l]{\rule{1cm}{2pt}}
\par\addvspace{1cm}
\noindent{\headerfont\Huge{\thetitle}}
\par
{\headerfont\Large{SE05-CCECAA: Cloud Computing and\newline Edge-Cloud Adaptive Architectures}}
\vskip\baselineskip
\noindent
\vfill
\noindent{\theauthor}
\par\addvspace{.5cm}
\noindent{\thedate}
\end{titlepage}


\pagecolor{white}



\subsection*{Context}
Netcompany is an IT consultancy service company based in Denmark. The company focuses on digital transformation and has acquired other companies to expand its portfolio and expertise. It has over 6,500 employees and aims to build flexible, scalable, and secure digital platforms for customers in both the public sector and private sector (often acquired by bidding contracts).

\subsection*{The Specified Problem}
Netcompany did present case specific.

The case is a Software-as-a-Service product called Airhart in a joint-venture between Netcompany and Copenhagen Airports called Smarter Airports. Netcompany has the IT technical knowledge and Copenhagen Airports possess the domain specific knowledge (airport management), and they have a commercial incentive (along with an effiency incentive for Copenhagen Airports' perspective) to build a great product.


\subsection*{The solution methodology}
Netcompany works problem-oriented and believes in agnostics when it comes to the tech stack. They start by building a core platform at first using a bottom-up approach and thinks scalability and modularity within the platform as it should fit different airports and each airport's own differences. They use core principles that are commonly associated with the Cloud Native era such as component architecture using messaging and APIs to connect between services.

By having this approach, Netcompany continuously balance between the proprietary feature sets of each cloud provider (the vendor) and portable solutions (often open source) ensuring a well adoption of cloud native technologies.

\subsection*{Takeaways}
I think their approach on problem-first, toolings second are something, that could be used in a lot of existing projects at companies around the world. I had a talk with a fellow student, where at his job, they decided to go with Node.js to try it out for a project, that wasn't suited for the use case at all, and they are now using many ressources as a consequence of the technical debt this decision has followed.

In cloud, I do however think that you should choose your supplier as semi-permanent. No matter if you are using Kubernetes, Knative or other cloud agnostic toolings, there is always some vendor specific batteries included. For instance in Kubernetes on GCP, you will use Google Cloud's proprietary CRDs, the same for AKS (Azure) or ECS (AWS).


\end{document}
